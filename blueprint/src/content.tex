% In this file you should put the actual content of the blueprint.
% It will be used both by the web and the print version.
% It should *not* include the \begin{document}
%
% If you want to split the blueprint content into several files then
% the current file can be a simple sequence of \input. Otherwise It
% can start with a \section or \chapter for instance.

\section{Main}

\begin{lemma} \label{small_lemma}
\leanok

Let $Y$ be a set and $x \in Y$ some point in $Y$. Let $C \sub Y$ be a non-empty set satisfying $C \ne {x}$. Then there is some $y \in Y$ such that $y \ne x$ and $y \in C$.
\end{lemma}

\begin{proof}
\leanok 

We split into cases depending on whether $x \in C$.

\vs

Case 1: $x \not \in C$. Since $C$ is non-empty, there is some $y \in C$. 
We claim such a $y$ satisfies $y \ne x$ and $y \in C$. The latter is by definition.
For the former, if $y = x$, then that $y \in C$ implies $x \in C$, contradicting $x \not \in C$.

\vs

Case 2: $x \in C$. We show that, assuming the conclusion is false, it holds that $C = \{x\}$. 
To show $\{x\} \subseteq C$, it suffices to show $x \in C$, but we know this already. To show
$C \subseteq \{x\}$, we take a $y \in C$, and by assumption, since $y \in C$, we know $y = x$
and hence $y \in \{x\}$.
\end{proof}


\vsss

\begin{proposition} \label{UGinF}
\leanok
\uses{small_lemma}
Let $\mc{F}$ be a finite union-closed family of sets and $\mc{G} \sub \mc{F}$ a non-empty sub-family. Then , $\bigcup \mc{G} \in \mc{F}$. Here, $\bigcup G$ is the union of all sets in $\mathcal{G}$. 
\end{proposition}

\begin{proof}
\leanok
We induct on the size of $\mc{G}$, with the base case of $|\mc{G}| = 0$ being vacuous, since $\mc{G}$ is non-empty. Now assume the result holds for all such $\mc{G}$ of size $|\mc{G}| = n$, and take such a $\mc{G}$ of size $|\mc{G}| = n+1$. 

\vs

We split into cases based on whether $G$ is a singleton.
\end{proof}





% \begin{theorem}[Smale 1958]
%     \label{thm:sphere_eversion}
%     \lean{sphere_eversion}
%     \leanok
%     \uses{def:immersion}
%     There is a homotopy of immersions of $𝕊^2$ into $ℝ^3$ from the inclusion map to
%     the antipodal map $a : q ↦ -q$.
%   \end{theorem}
    
%   \begin{proof}
%     \leanok
%     \uses{thm:open_ample, lem:open_ample_immersion}
%     This obviously follows from what we did so far.
%   \end